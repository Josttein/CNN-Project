\startsection[title={\em Convolutional} Neural Networks]
A convolutional neural network is an evolution of a classical multilayer percepetron network.
Recall the basic principle underpinning how a normal neuron in a neural network is supposed to work.
The neuron is supposed to \quote{look} for features in its input data.
If the neuron \quote{thinks} that those features are present in the input data it \quote{fires}.
Otherwise the neuron does not fire.

In a classical multilayer percepetron network this is implemented in the following way.
Each neuron contains a vector of weights, a bias and an activation function.
The input to the neuron---which must be a vector of equal length to the neuron's own weight vector---is combined with the neuron's weight vector using the dot product.
The neuron's bias is added onto the result which in turn is passed to the activation function which determines if the neuron \quote{fires} or not.

Using several layers of neurons one can achive quite remarkable results using this implementation of a neural network.
However, a multilayer percepetron is inherently limited.
The major problem is that neurons in these kinds of networks only accept input that is in the form of a vector.
This means that for applications where it is not natural for the input to be in a vector format, say image recognition, the input first has to be translated to a vector format.
Usually this results in a loss of information contained in the input.
In the typical case of image recongition, the input is in the form of one or more arrays of two dimentions.
For a multilayer percepetron to treat this input, the images has to be \quote{flattend} into a vector of one dimention before it can be passed on to the network.
This procedure eliminates some of the pixel relations in the image.
To deduce this, consider the process of reconstructing a flattend image.
If the image's dimentions prior to being flattend is not known, it is impossible, without the aid of pattern recognition, to reconstruct the image and be sure the reconstruction is equal to the original image.

To remedy this problem, there is a simple solution.
Instead of having the neuron contain a vector of weights, let it have an array of weights.
Changing the neuron's vector of weights into an array of weights also necessitates a change in the operation used to combine the weights with the input (which in a multilayer percepetron is the dot product).
There are two things to consider here.
The purpose of the weights is to look for features or {\em patterns} in the input and the operation must reflect this purpose of the weights.
Furthermore, the result of the operation should be a single number which, in a sense, representes the neurons \quote{initial} confidence that the feature is looking for is present in the input.
The operation which does both of these things is the {\em Hadamard product}.
The Hadamard product can be viewed as an extension of the dot product to two dimentional arrays.
It combines two arrays---of the same dimentions---by multiplying corresponding entries together and summing the results.
Which is precisly what the dot product does with two vectors.

We can even take this a step further to allow the neuron to treat inputs of not only a single two dimentional array but several two dimentional arrays.
A typical example where the input would consists of several interlinked two dimentional arrays is RGB images.
An RGB image consists of three arrays of pixel values (numbers) that desribe how red, green and blue an image is in each pixel.
The number of two dimentional arrays present in the input is known as the number of {\em channels} the input has.
In order for our neuron to treat inputs with more than one channel we let the neuron have as many channels as the input.
That is to say, wwe equip the neuron with as many weight arrays as there are channels in the input.
The weight array in each channel is combined with the inputs array in the same channel using the Hadamard product.
The results of these individual Hadamard products is then combined to form the \quote{final} Hadamard product---the neurons \quote{initial} confidence that the feature is looking for is present in the input.

A neural network which makes use of layers of neurons of this kind, is a convolutional neural network.
Why it is called a convolutional neural network will be explained in the next section.


Using this slightly more complicated implementation of a neuron, the neural network is able to treat inputs of one more two dimentional arrays directly (for each array present in the )
A network which contains layers of these kinds of neurons, is a convolutional neural network.


A convolutional neural network implements the very same idea, a neuron which looks for features and decides to fire, but with a more complicated implementation.
The benefit of this more complicated implementation is that it allows the network to preserve spatial information in its input.
Consider again a normal multilayer percepetron.
These networks work with inputs that are vectors of numbers.
In order for these networks to accept inputs that are multi-dimentional arrays, rather than one-dimentional vectors, the input first has to be \quote{flattened} before it is passed on to the network.
Using the MINST database of handwritten digits stored as $28{\rm x}28$ grayscale images as an example, for a multilayer percepetron to process such an input the image first has to be flattend to a $28 \cdot 28=784$ long vector.

A \quote{convolutional neuron} does not have this limitation, it can treat a multidimensional input directly.
The idea is to replace the simple neuron of a multilayer percepetron---made up of a weight vector and a bias---with a \quote{kernel} or \quote{filter}.
The kernel contains a multidimensional array of weights, instead of a simple vector, and a bias.
Instead of \quote{looking} at a vector of inputs, the kernel can look at an array of inputs which allows it to \quote{look} for features that can only be properly described by a full array rather than a flattened vector.
Necessarly, such a \quote{convolutional neuron} requires a different mathematical operation to do the \quote{looking for feature} job of the neuron.
The operation that is used is a convolution hence the name, convolutional neural network.

\startsubsection[title=The convolutional layer]
A convolutional layer consists of three parts: the input (a multidimensional array), the kernel (also a multidimensional array with a bias term) and the output which is called a {\em feature map}.
The layers forward operation is
\startformula
{\rm feature\ map} = {\rm Convolution}\,({\rm input}, {\rm kernel}) + {\rm bias}
\stopformula
Figure~\in[conv-operation] on page~\at[conv-operation] illustrates how the feature map is calculated using the kernel and a input.
The kernel \quote{scans} the input and produces a number based upon how confident it is that a certain feature is present at the area it is looking at.
The bias is added onto this number which produces the final number placed in the feature map.

\startplacefigure[reference=conv-operation,
                   title={The basic forward operation of a convolutional layer},
                   location=top]
\startcombination[3*3]
{\externalfigure[./Images/conv-0.jpg][width=.3\textwidth]}{}
{\externalfigure[./Images/conv-1.jpg][width=.3\textwidth]}{}
{\externalfigure[./Images/conv-2.jpg][width=.3\textwidth]}{}
{\externalfigure[./Images/conv-3.jpg][width=.3\textwidth]}{}
{\externalfigure[./Images/conv-4.jpg][width=.3\textwidth]}{}
{\externalfigure[./Images/conv-5.jpg][width=.3\textwidth]}{}
{\externalfigure[./Images/conv-6.jpg][width=.3\textwidth]}{}
{\externalfigure[./Images/conv-7.jpg][width=.3\textwidth]}{}
{\externalfigure[./Images/conv-8.jpg][width=.3\textwidth]}{}
\stopcombination
\stopplacefigure



In the case where the input has multiple channels---as is the case in Figure~\in[conv-operation]---each input is a sequence of arrays.
The kernel has to an have equal amount of channels as its input.
Each channel is scanned seperately and their values are added together.

Let's us derive the proper mathematical formulation for calculating the feature map.
The feature map is a two dimentional array which we will denote $F$, the individual entries are dentoed $f(i,j)$ where $i$ is the row and $j$ the column.
The kernel is a multidimensional array which we will denote $K$, the individual entries are denoted $K(i,j,c)$ where $c$ is the channel.
The input is, same as the kernel, a multidimensional array which we will denote $M$, the individual entries are denoted $M(i,j,c)$.
The horisontal lengths and vertical lengths of the kernel and input will be denoted using a $h$ and $v$ prefixes (e.g.\ $hK$ is the horisontal length of the kernel).
A typical MINST image of dimentions $28{\rm x}28$ will have $hM = vM = 28$.
The number of channels in the input and kernel will be dentoed by $nC$.
The bias term is denoted $b$.
As we can see from Figure~\in[conv-operation], the formula for an individual entry in the feature map is

\startformula
f(i',j') = \sum_{c = 1}^{nC} \sum_{j = 1}^{vK} \sum_{i = 1}^{hK} \Bigl( K(i,j,c) \cdot M(i' + i,j' + j, c) \Bigr) + b
\stopformula
\stopsubsection

Go on \dots

\startsubsection[title=Multiple channels]
\stopsubsection

\startsubsubsection[title=Altering the stride]
What the stride is and the effects of altering it.
\stopsubsubsection

\startsubsubsection[title=Zero padding]
\stopsubsubsection

\startsubsection[title=The pooling layer]
{\em Downsampling} and purpose of downsampling.
Noise reduction and computational reduction.
\stopsubsection

\startsubsection[title=Backward Propagation]
Obtain formula for derivative of convolution and pooling.
\stopsubsection
\stopsection