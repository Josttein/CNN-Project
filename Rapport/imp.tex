\startsection[title=Python Implementation]
Knwon that we have taken a look at the theory behind convolutional neural networks, let us look at an implementation of one.
The nerual network we are going to be looking at is one written by 
Alejandro Escontrela.
It is written in Python using the NumPy library and is publicly avaliable at \url[github]

The network is going to tackle the classic neural network problem.
Categorizing the handwritten digits in the MNIST database.
Compared to other problems convolutional neural networks are commonly faced with, categorizing the MNIST images is rather simple.
So the network will use a comperativly simple architecture as shown in Figure~\in[net-architecture] on page~\at[net-architecture].
The network uses no zero padding.
The first two layer consists of two indentical convolutional layers.
They are in turn follwed by a single max-polling layer which notably, uses a stride of 2.
The output is then flattened and passed to a dense layer with 128 neurons.
The output of this layer is passed to the final dense layer, which necessarly consists of 10 neurons.

\startplacefigure[reference=net-architecture,
                   title={The architecture of the network},
                   location=top]
\externalfigure[Images/conv-net-alej.png]
\stopplacefigure

\startsubsection[title=Convolution function]
Here is the python code for the forward convolution operation of the network
\starttyping
def convolution(image, filt, bias, s=1):
    '''
    Confolves `filt` over `image` using stride `s`
    '''
    (n_f, n_c_f, f, _) = filt.shape # filter dimensions
    n_c, in_dim, _ = image.shape # image dimensions
    
    out_dim = int((in_dim - f)/s)+1 # calculate output dimensions
    
    assert n_c == n_c_f, "Dimensions of filter must match dimensions of input image"
    
    out = np.zeros((n_f,out_dim,out_dim))
    
    # convolve the filter over every part of the image, adding the bias at each step. 
    for curr_f in range(n_f):
        curr_y = out_y = 0
        while curr_y + f <= in_dim:
            curr_x = out_x = 0
            while curr_x + f <= in_dim:
                out[curr_f, out_y, out_x] = np.sum(filt[curr_f] * image[:,curr_y:curr_y+f, curr_x:curr_x+f]) + bias[curr_f]
                curr_x += s
                out_x += 1
            curr_y += s
            out_y += 1
        
    return out
\stoptyping
\stopsubsection

\startsubsection[title=Max pooling]
Here is the python code for the forward max-pooling operation of the network
\stopsubsection

\startsubsection[title=Various functions]
The network uses ReLU for its activation function, 

\stopsubsection





\starttyping
def maxpool(image, f=2, s=2):
    '''
    Downsample `image` using kernel size `f` and stride `s`
    '''
    n_c, h_prev, w_prev = image.shape
    
    h = int((h_prev - f)/s)+1
    w = int((w_prev - f)/s)+1
    
    downsampled = np.zeros((n_c, h, w))
    for i in range(n_c):
        # slide maxpool window over each part of the image and assign the max value at each step to the output
        curr_y = out_y = 0
        while curr_y + f <= h_prev:
            curr_x = out_x = 0
            while curr_x + f <= w_prev:
                downsampled[i, out_y, out_x] = np.max(image[i, curr_y:curr_y+f, curr_x:curr_x+f])
                curr_x += s
                out_x += 1
            curr_y += s
            out_y += 1
    return downsampled
\stoptyping

\starttyping
def convolutionBackward(dconv_prev, conv_in, filt, s):
    '''
    Backpropagation through a convolutional layer. 
    '''
    (n_f, n_c, f, _) = filt.shape
    (_, orig_dim, _) = conv_in.shape
    ## initialize derivatives
    dout = np.zeros(conv_in.shape) 
    dfilt = np.zeros(filt.shape)
    dbias = np.zeros((n_f,1))
    for curr_f in range(n_f):
        # loop through all filters
        curr_y = out_y = 0
        while curr_y + f <= orig_dim:
            curr_x = out_x = 0
            while curr_x + f <= orig_dim:
                # loss gradient of filter (used to update the filter)
                dfilt[curr_f] += dconv_prev[curr_f, out_y, out_x] * conv_in[:, curr_y:curr_y+f, curr_x:curr_x+f]
                # loss gradient of the input to the convolution operation (conv1 in the case of this network)
                dout[:, curr_y:curr_y+f, curr_x:curr_x+f] += dconv_prev[curr_f, out_y, out_x] * filt[curr_f] 
                curr_x += s
                out_x += 1
            curr_y += s
            out_y += 1
        # loss gradient of the bias
        dbias[curr_f] = np.sum(dconv_prev[curr_f])
    
    return dout, dfilt, dbias
\stoptyping

\starttyping
def maxpoolBackward(dpool, orig, f, s):
    '''
    Backpropagation through a maxpooling layer. The gradients are passed through the indices of greatest value in the original maxpooling during the forward step.
    '''
    (n_c, orig_dim, _) = orig.shape
    
    dout = np.zeros(orig.shape)
    
    for curr_c in range(n_c):
        curr_y = out_y = 0
        while curr_y + f <= orig_dim:
            curr_x = out_x = 0
            while curr_x + f <= orig_dim:
                # obtain index of largest value in input for current window
                (a, b) = nanargmax(orig[curr_c, curr_y:curr_y+f, curr_x:curr_x+f])
                dout[curr_c, curr_y+a, curr_x+b] = dpool[curr_c, out_y, out_x]
                
                curr_x += s
                out_x += 1
            curr_y += s
            out_y += 1
        
    return dout
\stoptyping

\startitemize
\startitem 
The problem the network is going to tackle (classic MNIST image recognition)
\stopitem
\startitem
The architecture of the network
\stopitem
\startitem
Choice of various functions and their associated code
\stopitem
\startitem
Results with time performance
\stopitem
\startitem
Possible Modifications
\stopitem
\stopitemize
\stopsection