\startsection[title=Layers of convolutional neural networks]

In this section we are going to look at the fundamental equations which govern the operation of each layer of a convolutional neural network.
The equations we are going to be looking at are primarily the \quote{lowest level} ones, relating each number from the previous layer to each number in the next layer.

\startsubsubject[title=Terminology and notation]
Before we get started we need to introduce some terms and notation.
Scalars will be represented with a lowercase letter (e.g.\ $x$).
Vectors will be represented with a bold lowercase letter (e.g.\ $\bi x$).
Matricies will be represented with an uppercase letter (e.g.\ $X$).
Vectors of matricies---vectors where each entry is a matrix---will be represented with a bold uppercase letter (e.g.\ $\bi X$).
This scheme has the added benefit that for the most part, aggregates of a layer will be represented in bold while data in individual channels will be represented in normal style letters.

Speaking of layers we need some symbols for the various parts that make up a layer.
Firstly we will need symbols for the two principle components of a neuron.
Namely its weights and its bias.
Weights will be represented with the letter \quote{w} while the bias will be represented with the letter \quote{b}.
The preliminary output of a neruon (or layer of neurons), before it is passed through an activation function, will be represented with the letter \quote{z}.
The output after activation will be represented with the letter \quote{a}.
Where does one layer end and the next one begin?
Everything after the previous layer's activation up until and including the current layer's activation constitues the layer in question.

We need the ability to reference specific positions in each layer and for that we need {\em a lot} of indexes.
Since we want our mathematics to match our code, to a certain extent, all the indices used in this paper will start at $0$.
The layer index which tells us which layer we are looking at will be written using a parenthetical superscript (e.g.\ ${\bi a}^{(l)}$ is the activation vector of layer $l$).
The channel index which tells us which channel we are looking at will be written using a parenthetical subscript (e.g.\ $a^{(l)}_{(c)}$ is the scalar activation of layer $l$ in channel $c$).
The output of each neuron in a layer becomes a channel in the next layer.
To reference both the input and output channel, a double parenthetical subscript will be used (e.g.\ $w^{(l)}_{(c,c')}$ is the scalar weight in layer $l$ connecting channel $c$ from layer $l$ to channel $c'$ in layer $l-1$).
To reference positions within a matrix we will use regular xy-coordinates (e.g.\ $a(x,y)$ is the scalar in row $x$ at column $y$).
Equations which relate a previous layer to the next one will generally use blue color to makr indicies belonging to the previous layer.
Red color will be used to mark indices beloning to the next layer.

We will often need to know the {\em dimensions} of layer.
How many channels does it have and what are the sizes of the matrices in that layer.
These dimensions will be written using the greek letter eta ($\eta$) with a layer index and a subscript to identify which dimention is expressed.
A \quote{c} subscript means it is the number of channels in the layer {\em minus one}.
A \quote{x} and \quote{y} subscript means it is row and column length {\em minus one} respectivly.
We use the letter $\eta$ instead of $n$ to remind the reader that all dimensions are substracted 1, because we want all our indices to start at zero.
As an example, if layer $3$ has eight individual channels then $\eta^{(3)}_c = 7$.

The inital input to a network can be though of as the first activation to the network.
So we are going to denote the inital input using the appropriate \quote{a} letter with a zero in parenthetical subscript.
\stopsubsubject


\startsubsection[title=Fully-connected layer]
\input 2/fc-layer.tex
\stopsubsection

\startsubsection[title=Convolutional layer]
\input 2/cv-layer.tex % TODO

\startsubsubsection[title=Where is the convolution?]
\input 2/where.tex % TODO
\stopsubsubsection
\stopsubsection

\startsubsection[title=Downsampling layer]
\input 2/dv-layer.tex
\stopsubsection

\startsubsection[title=The activation function]
\input 2/activ.tex
\stopsubsection

\startsubsection[title=Output layer]
\input 2/out.tex
\stopsubsection