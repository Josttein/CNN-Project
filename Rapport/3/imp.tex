\startsection[title=Python Implementation]
Knwon that we have taken a look at the theory behind convolutional neural networks, let us look at an implementation of one.
The nerual network we are going to be looking at is one written by 
Alejandro Escontrela.
It is written in Python using the NumPy library and is publicly avaliable at \url[github]

The network is going to tackle the classic neural network problem.
Categorizing the handwritten digits in the MNIST database.
Compared to other problems convolutional neural networks are commonly faced with, categorizing the MNIST images is rather simple.
So the network will use a comperativly simple architecture as shown in Figure~\in[net-architecture] on page~\at[net-architecture].
The network uses no zero padding.
The first two layer consists of two indentical convolutional layers.
They are in turn follwed by a single max-polling layer which notably, uses a stride of 2.
The output is then flattened and passed to a dense layer with 128 neurons.
The output of this layer is passed to the final dense layer, which necessarly consists of 10 neurons.

\startplacefigure[
    reference=net-architecture,
    title={The architecture of the network},
    location=bottom,
]
\externalfigure[Images/conv-net-alej.png]
\stopplacefigure

\startsubsection[title=Convolution function]
Here is the python code for the forward convolution operation of the network
\starttyping
def convolution(image, filt, bias, s=1):
    '''
    Confolves `filt` over `image` using stride `s`
    '''
    (n_f, n_c_f, f, _) = filt.shape # filter dimensions
    n_c, in_dim, _ = image.shape # image dimensions
    
    out_dim = int((in_dim - f)/s)+1 # calculate output dimensions
    
    assert n_c == n_c_f, "Dimensions of filter must match dimensions of input image"
    
    out = np.zeros((n_f,out_dim,out_dim))
    
    # convolve the filter over every part of the image, adding the bias at each step. 
    for curr_f in range(n_f):
        curr_y = out_y = 0
        while curr_y + f <= in_dim:
            curr_x = out_x = 0
            while curr_x + f <= in_dim:
                out[curr_f, out_y, out_x] = np.sum(filt[curr_f] * image[:,curr_y:curr_y+f, curr_x:curr_x+f]) + bias[curr_f]
                curr_x += s
                out_x += 1
            curr_y += s
            out_y += 1
        
    return out
\stoptyping
\stopsubsection

\startsubsection[title=Max pooling]
Here is the python code for the forward max-pooling operation of the network
\stopsubsection

\startsubsection[title=Various functions]
The network uses ReLU for its activation function, 

\stopsubsection





\starttyping
def maxpool(image, f=2, s=2):
    '''
    Downsample `image` using kernel size `f` and stride `s`
    '''
    n_c, h_prev, w_prev = image.shape
    
    h = int((h_prev - f)/s)+1
    w = int((w_prev - f)/s)+1
    
    downsampled = np.zeros((n_c, h, w))
    for i in range(n_c):
        # slide maxpool window over each part of the image and assign the max value at each step to the output
        curr_y = out_y = 0
        while curr_y + f <= h_prev:
            curr_x = out_x = 0
            while curr_x + f <= w_prev:
                downsampled[i, out_y, out_x] = np.max(image[i, curr_y:curr_y+f, curr_x:curr_x+f])
                curr_x += s
                out_x += 1
            curr_y += s
            out_y += 1
    return downsampled
\stoptyping

\starttyping
def convolutionBackward(dconv_prev, conv_in, filt, s):
    '''
    Backpropagation through a convolutional layer. 
    '''
    (n_f, n_c, f, _) = filt.shape
    (_, orig_dim, _) = conv_in.shape
    ## initialize derivatives
    dout = np.zeros(conv_in.shape) 
    dfilt = np.zeros(filt.shape)
    dbias = np.zeros((n_f,1))
    for curr_f in range(n_f):
        # loop through all filters
        curr_y = out_y = 0
        while curr_y + f <= orig_dim:
            curr_x = out_x = 0
            while curr_x + f <= orig_dim:
                # loss gradient of filter (used to update the filter)
                dfilt[curr_f] += dconv_prev[curr_f, out_y, out_x] * conv_in[:, curr_y:curr_y+f, curr_x:curr_x+f]
                # loss gradient of the input to the convolution operation (conv1 in the case of this network)
                dout[:, curr_y:curr_y+f, curr_x:curr_x+f] += dconv_prev[curr_f, out_y, out_x] * filt[curr_f] 
                curr_x += s
                out_x += 1
            curr_y += s
            out_y += 1
        # loss gradient of the bias
        dbias[curr_f] = np.sum(dconv_prev[curr_f])
    
    return dout, dfilt, dbias
\stoptyping

\starttyping
def maxpoolBackward(dpool, orig, f, s):
    '''
    Backpropagation through a maxpooling layer. The gradients are passed through the indices of greatest value in the original maxpooling during the forward step.
    '''
    (n_c, orig_dim, _) = orig.shape
    
    dout = np.zeros(orig.shape)
    
    for curr_c in range(n_c):
        curr_y = out_y = 0
        while curr_y + f <= orig_dim:
            curr_x = out_x = 0
            while curr_x + f <= orig_dim:
                # obtain index of largest value in input for current window
                (a, b) = nanargmax(orig[curr_c, curr_y:curr_y+f, curr_x:curr_x+f])
                dout[curr_c, curr_y+a, curr_x+b] = dpool[curr_c, out_y, out_x]
                
                curr_x += s
                out_x += 1
            curr_y += s
            out_y += 1
        
    return dout
\stoptyping

\starttyping
def conv(image, label, params, conv_s, pool_f, pool_s):
    
    [f1, f2, w3, w4, b1, b2, b3, b4] = params 
    
    ################################################
    ############## Forward Operation ###############
    ################################################
    conv1 = convolution(image, f1, b1, conv_s) # convolution operation
    conv1[conv1<=0] = 0 # pass through ReLU non-linearity
    
    conv2 = convolution(conv1, f2, b2, conv_s) # second convolution operation
    conv2[conv2<=0] = 0 # pass through ReLU non-linearity
    
    pooled = maxpool(conv2, pool_f, pool_s) # maxpooling operation
    
    (nf2, dim2, _) = pooled.shape
    fc = pooled.reshape((nf2 * dim2 * dim2, 1)) # flatten pooled layer
    
    z = w3.dot(fc) + b3 # first dense layer
    z[z<=0] = 0 # pass through ReLU non-linearity
    
    out = w4.dot(z) + b4 # second dense layer
     
    probs = softmax(out) # predict class probabilities with the softmax activation function
    
    ################################################
    #################### Loss ######################
    ################################################
    
    loss = categoricalCrossEntropy(probs, label) # categorical cross-entropy loss
        
    ################################################
    ############# Backward Operation ###############
    ################################################
    dout = probs - label # derivative of loss w.r.t. final dense layer output
    dw4 = dout.dot(z.T) # loss gradient of final dense layer weights
    db4 = np.sum(dout, axis = 1).reshape(b4.shape) # loss gradient of final dense layer biases
    
    dz = w4.T.dot(dout) # loss gradient of first dense layer outputs 
    dz[z<=0] = 0 # backpropagate through ReLU 
    dw3 = dz.dot(fc.T)
    db3 = np.sum(dz, axis = 1).reshape(b3.shape)
    
    dfc = w3.T.dot(dz) # loss gradients of fully-connected layer (pooling layer)
    dpool = dfc.reshape(pooled.shape) # reshape fully connected into dimensions of pooling layer
    
    dconv2 = maxpoolBackward(dpool, conv2, pool_f, pool_s) # backprop through the max-pooling layer(only neurons with highest activation in window get updated)
    dconv2[conv2<=0] = 0 # backpropagate through ReLU
    
    dconv1, df2, db2 = convolutionBackward(dconv2, conv1, f2, conv_s) # backpropagate previous gradient through second convolutional layer.
    dconv1[conv1<=0] = 0 # backpropagate through ReLU
    
    dimage, df1, db1 = convolutionBackward(dconv1, image, f1, conv_s) # backpropagate previous gradient through first convolutional layer.
    
    grads = [df1, df2, dw3, dw4, db1, db2, db3, db4] 
    
    return grads, loss
\stoptyping

\startsection[title=Batch Optimization]
When you have a large training set, it can be time consuming to evaluate a cost function on the data set in its entirety.
A common way to avoid this is by instead evaluating the cost function in batches, or subsets of the training data.
In other words, for each batch, we evaluate the cost function and adjust the paramaters accordingly. 
After using all the training data to tweak the paramaters, we have finished an epoch.
This process can then be repeated over several epochs, but in our case, we have found that two epochs works sufficiently.
The fact that we adjust the paramaters after each batch, means that we have to choose a suitable optimization algorithm such as Stochastic Gradient Descent.
In this implementation, however, the author has used an algorithm that is shown to be very efficent in terms of batch optimization, namely the Adam Gradient Descent algorithm.
We will not spend time explaining how it works, but its main principles are very similar to that of the classic gradient descent.
For each training data in the batch, we evaluate the output of the loss by the forward opartion and retrieve the gradients using the backward operation described earlier. 
Afterwards, we evaluate the cost function of the batch and adjust the parameters.
The implementation of this concept in python, is summarized in figure ()



  

\stopsection

\starttyping
def adamGD(batch, num_classes, lr, dim, n_c, beta1, beta2, params, cost):
    '''
    update the parameters through Adam gradient descnet.
    '''
    [f1, f2, w3, w4, b1, b2, b3, b4] = params
    
    X = batch[:,0:-1] # get batch inputs
    X = X.reshape(len(batch), n_c, dim, dim)
    Y = batch[:,-1] # get batch labels
    
    cost_ = 0
    batch_size = len(batch)
    
    # initialize gradients and momentum,RMS params

    for i in range(batch_size):
        
        x = X[i]
        y = np.eye(num_classes)[int(Y[i])].reshape(num_classes, 1) # convert label to one-hot
        
        # Collect Gradients for training example
        grads, loss = conv(x, y, params, 1, 2, 2)
        [df1_, df2_, dw3_, dw4_, db1_, db2_, db3_, db4_] = grads
        
        df1+=df1_
        db1+=db1_
        df2+=df2_
        db2+=db2_
        dw3+=dw3_
        db3+=db3_
        dw4+=dw4_
        db4+=db4_

        cost_+= loss

    # Parameter Update  

    

    cost_ = cost_/batch_size
    cost.append(cost_)

    params = [f1, f2, w3, w4, b1, b2, b3, b4]
    
    return params, cost


\stoptyping

\starttyping
def train(num_classes = 10, lr = 0.01, beta1 = 0.95, beta2 = 0.99, img_dim = 28, img_depth = 1, f = 5, num_filt1 = 8, num_filt2 = 8, batch_size = 32, num_epochs = 2, save_path = 'params.pkl'):

    # training data
    m =50000
    X = extract_data('train-images-idx3-ubyte.gz', m, img_dim)
    y_dash = extract_labels('train-labels-idx1-ubyte.gz', m).reshape(m,1)
    X-= int(np.mean(X))
    X/= int(np.std(X))
    train_data = np.hstack((X,y_dash))
    
    np.random.shuffle(train_data)

    ## Initializing all the parameters
    f1, f2, w3, w4 = (num_filt1 ,img_depth,f,f), (num_filt2 ,num_filt1,f,f), (128,800), (10, 128)
    f1 = initializeFilter(f1)
    f2 = initializeFilter(f2)
    w3 = initializeWeight(w3)
    w4 = initializeWeight(w4)

    b1 = np.zeros((f1.shape[0],1))
    b2 = np.zeros((f2.shape[0],1))
    b3 = np.zeros((w3.shape[0],1))
    b4 = np.zeros((w4.shape[0],1))

    params = [f1, f2, w3, w4, b1, b2, b3, b4]

    cost = []

    print("LR:"+str(lr)+", Batch Size:"+str(batch_size))

    for epoch in range(num_epochs):
        np.random.shuffle(train_data)
        batches = [train_data[k:k + batch_size] for k in range(0, int(train_data.shape[0]/16), batch_size)]

        t = tqdm(batches)
        for x,batch in enumerate(t):
            params, cost = adamGD(batch, num_classes, lr, img_dim, img_depth, beta1, beta2, params, cost)
            t.set_description("Cost: %.2f" % (cost[-1]))
            
    to_save = [params, cost]
    
    with open(save_path, 'wb') as file:
        pickle.dump(to_save, file)
        
    return cost
\stoptyping
\startsection[title=Results]
Using our own laptops, we initially trained the network on the entire MNIST data set, which is in fact 10.000 labeled images of handwritten digits.
The results are are represented on figure ().
Though the network's predictions were accurate, we found that the eight-hour total run-time of the training phase  was too long.
To remedy this, we simply reduced the training set by a factor of 2 then by a factor of 16, respectively achieving a run-time of 4 hours and then 30 minutes. 





\stopsection

\startitemize
\startitem 
The problem the network is going to tackle (classic MNIST image recognition)
\stopitem
\startitem
The architecture of the network
\stopitem
\startitem
Choice of various functions and their associated code
\stopitem
\startitem
Results with time performance
\stopitem
\startitem
Possible Modifications
\stopitem
\stopitemize
\stopsection
